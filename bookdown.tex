\documentclass[]{ctexbook}
\usepackage{lmodern}
\usepackage{amssymb,amsmath}
\usepackage{ifxetex,ifluatex}
\usepackage{fixltx2e} % provides \textsubscript
\ifnum 0\ifxetex 1\fi\ifluatex 1\fi=0 % if pdftex
  \usepackage[T1]{fontenc}
  \usepackage[utf8]{inputenc}
\else % if luatex or xelatex
  \ifxetex
    \usepackage{xltxtra,xunicode}
  \else
    \usepackage{fontspec}
  \fi
  \defaultfontfeatures{Ligatures=TeX,Scale=MatchLowercase}
\fi
% use upquote if available, for straight quotes in verbatim environments
\IfFileExists{upquote.sty}{\usepackage{upquote}}{}
% use microtype if available
\IfFileExists{microtype.sty}{%
\usepackage{microtype}
\UseMicrotypeSet[protrusion]{basicmath} % disable protrusion for tt fonts
}{}
\usepackage[b5paper,tmargin=2.5cm,bmargin=2.5cm,lmargin=3.5cm,rmargin=2.5cm]{geometry}
\usepackage[unicode=true]{hyperref}
\PassOptionsToPackage{usenames,dvipsnames}{color} % color is loaded by hyperref
\hypersetup{
            pdftitle={狗熊会人才计划第三期------My case},
            pdfauthor={yimeng},
            colorlinks=true,
            linkcolor=Maroon,
            citecolor=Blue,
            urlcolor=Blue,
            breaklinks=true}
\urlstyle{same}  % don't use monospace font for urls
\usepackage{natbib}
\bibliographystyle{apalike}
\usepackage{longtable,booktabs}
% Fix footnotes in tables (requires footnote package)
\IfFileExists{footnote.sty}{\usepackage{footnote}\makesavenoteenv{long table}}{}
\IfFileExists{parskip.sty}{%
\usepackage{parskip}
}{% else
\setlength{\parindent}{0pt}
\setlength{\parskip}{6pt plus 2pt minus 1pt}
}
\setlength{\emergencystretch}{3em}  % prevent overfull lines
\providecommand{\tightlist}{%
  \setlength{\itemsep}{0pt}\setlength{\parskip}{0pt}}
\setcounter{secnumdepth}{5}
% Redefines (sub)paragraphs to behave more like sections
\ifx\paragraph\undefined\else
\let\oldparagraph\paragraph
\renewcommand{\paragraph}[1]{\oldparagraph{#1}\mbox{}}
\fi
\ifx\subparagraph\undefined\else
\let\oldsubparagraph\subparagraph
\renewcommand{\subparagraph}[1]{\oldsubparagraph{#1}\mbox{}}
\fi

% set default figure placement to htbp
\makeatletter
\def\fps@figure{htbp}
\makeatother

\usepackage{booktabs}
\usepackage{longtable}

\usepackage{framed,color}
\definecolor{shadecolor}{RGB}{248,248,248}

\renewcommand{\textfraction}{0.05}
\renewcommand{\topfraction}{0.8}
\renewcommand{\bottomfraction}{0.8}
\renewcommand{\floatpagefraction}{0.75}

\let\oldhref\href
\renewcommand{\href}[2]{#2\footnote{\url{#1}}}

\makeatletter
\newenvironment{kframe}{%
\medskip{}
\setlength{\fboxsep}{.8em}
 \def\at@end@of@kframe{}%
 \ifinner\ifhmode%
  \def\at@end@of@kframe{\end{minipage}}%
  \begin{minipage}{\columnwidth}%
 \fi\fi%
 \def\FrameCommand##1{\hskip\@totalleftmargin \hskip-\fboxsep
 \colorbox{shadecolor}{##1}\hskip-\fboxsep
     % There is no \\@totalrightmargin, so:
     \hskip-\linewidth \hskip-\@totalleftmargin \hskip\columnwidth}%
 \MakeFramed {\advance\hsize-\width
   \@totalleftmargin\z@ \linewidth\hsize
   \@setminipage}}%
 {\par\unskip\endMakeFramed%
 \at@end@of@kframe}
\makeatother

\makeatletter
\@ifundefined{Shaded}{
}{\renewenvironment{Shaded}{\begin{kframe}}{\end{kframe}}}
\@ifpackageloaded{fanyverb}{%
  % https://github.com/CTeX-org/ctex-kit/issues/331
  \RecustomVerbatimEnvironment{Highlighting}{Verbatim}{commandchars=\\\{\},formatcom=\xeCJKVerbAddon}%
}{}
\makeatother

\usepackage{makeidx}
\makeindex

\urlstyle{tt}

\usepackage{amsthm}
\makeatletter
\def\thm@space@setup{%
  \thm@preskip=8pt plus 2pt minus 4pt
  \thm@postskip=\thm@preskip
}
\makeatother

\frontmatter

\title{狗熊会人才计划第三期------My case}
\author{yimeng}
\date{2018-09-03}

\let\BeginKnitrBlock\begin \let\EndKnitrBlock\end
\begin{document}
\maketitle


\thispagestyle{empty}

\begin{center}
献给……

正经点儿!
\end{center}

\setlength{\abovedisplayskip}{-5pt}
\setlength{\abovedisplayshortskip}{-5pt}

{
\setcounter{tocdepth}{2}
\tableofcontents
}
\listoftables
\listoffigures
\chapter*{前言}


嗨,各位。我写了一本书。这本书是这样的,除了最后一章,前面都是一个一个的TASK,分别为题目和我自己的答案,有些章节还有狗熊会提供的标准答案;最后一章是Final
Project的内容。第 \ref{intro} 章介绍了啥啥,第 \ref{wind}
章说了啥啥,然后是啥啥\ldots{}\ldots{}

我用了两个 R 包编译这本书,分别是 \textbf{knitr}\index{knitr}
\citep{xie2015} 和 \textbf{bookdown}\index{bookdown}
\citep{R-bookdown}。

\section*{致谢}


非常感谢狗熊会各位老师,尤其是水妈和雪姨对我的帮助(催促),使得这些TASK顺利完成。也感谢yalei没有对我的智商放弃,才有了这样一本书。

\BeginKnitrBlock{flushright}
yimeng 于人民大学
\EndKnitrBlock{flushright}

\chapter*{作者简介}\label{author}


人生终极的三大梦想:

\begin{itemize}
\tightlist
\item
  在日本东京街头吃一碗拉面;
\item
  在马来西亚摘一个榴莲;
\item
  在北欧的某个小镇看一次极光;
\end{itemize}

最好实现的一个梦想: 学好统计,写好代码,先挣钱再说。

\mainmatter

\chapter{狗熊会人才计划介绍}\label{intro}

在第三期人才计划的线上开营仪式中,熊大分别对狗熊会和人才计划做了详细介绍,权威官方,因此直接照搬来这里。

\begin{quote}
以下来自熊大\&水妈。
\end{quote}

\section{狗熊会,what's that?!}\label{whats-that}

狗熊会?乍一听好像一个黑涩会,这是一个什么鬼组织?如果让我用一两句话去解释,还真的有点困难。如果大家有兴趣,可以在狗熊会公号输入关键词``前世今生'',你会听到一个关于狗熊会来龙去脉的详细解释。简单滴说,狗熊会是一个关注数据产业实践,关注数据科学教育的盈利组织。狗熊会必须是一个盈利组织,因为狗熊会没有任何资金支持,如果自己不能养活自己,第二天就要死翘翘。但是,狗熊会的使命却一点也不盈利,更像一个非盈利组织。狗熊会的使命两句话:\textbf{聚数据英才、助产业振兴}。

第一句``聚数据英才''讲的是狗熊会,之于数据科学教育,的情怀。狗熊会的绝大多数联合创始人都是优秀的老师,他们希望通过狗熊会的平台,让更多的同学(例如:你们),可以享受到更好的数据科学教育。

第二句``助产业振兴''讲的是狗熊会,之于数据科学研究,的理念。狗熊会的绝大多数联合创始人也是优秀的学者,或者是企业家。我们有一个共同的信念,我们不相信,数据科学的真知灼见,能够在办公室里,在笔记本上被敲打出来。我们相信,最优秀的数据科学研究,一定是伴随着数据产业的发展进步,伴随他痛并快乐的成长。这是一个令人激动的进程,我们希望介入其中,而不做旁观者。

\section{人才计划,OMD!}\label{omd}

狗熊会使命的第一句话``聚数据英才'',如何聚?怎么聚?狗熊会有很多种方式。首先,狗熊会的主要创始人全部都是优秀的老师,我们认真上好每一堂课,给尽可能多的同学传授靠谱的数据科学之,这就是最基础的``聚数据英才的方式''。但是,这样并不能惠及更多的同学。例如,如果大家想听熊大的现场课程,你就必须是北大的学生,这是很挑战的,很不容易。但是,其他的非北大的学生,也想跟熊大学习朴素的数据价值观,肿么办?因此,我们有了第二种方式:数据科学研习社。数据科学研习社是,基于高校的,学生自发组织的,数据科学自学研习组织。狗熊会老师会亲自带大家入门,通过一系列精心设计的TASK,帮助大家培养自主学习的习惯,确定学习研究的良好方式。所以,这是研习社,是狗熊会``聚数据英才''的第二种方式。但是,这还不够。研习社要求很低,基本上只要同学认真努力,你就可以毕业了。对于部分自我要求更高的同学,需要一种更加严格,更加挑战的学习经历,这就是``人才计划''。研习社以学校为单位,集体形式参加。而人才计划全靠个人。所以,这是狗熊会最看重的数据科学人才培养计划,而且是100\%的公益项目。我希望通过这个计划的学习,``聚数据英才,助产业振兴''的理念,能够像DNA一样,也跟植入你们的DNA。伴随你们的学习进步,幸福成长。

更具体而言,我们对人才计划的合格毕业生有三点预期。

第一、你要好学。请你记住,人才计划 \textbf{什么都给不了你}
。给不了你文凭,给不了你奖金。为你能给你的是:学习知识的机会。所以,只有非常纯粹的好学之心,才能让你坚持下来。

第二、你要努力。狗熊会人才计划是每2-3天一个艰巨的TASK会布置下来。而且,
\textbf{没人教你} 怎么做,请help
yourself。狗熊会人才计划的目标是最优秀的学生,最优秀的学生,显然不是
\textbf{教} 出来的,都是自己 \textbf{练}
出来的。所以,狗熊会的整个教育理念都是 \textbf{练} ,而不是 \textbf{教}
。你要非常努力,这里的学习挑战强度,会远远超出你平时的课程学习。

第三、你要谦卑。狗熊会坚持一个最朴素的教育理念。你是学生,就好好当学生,不许评价老师。这样,老师才敢用最严格的标准要求你。所以,在人才计划学习,被批评是常态,被表扬是变态(很少)。因此,你需要一个强大的内心,面对批评,面对挫折,面对挑战,面对失败,面对被淘汰。

对以上三点任何一点有不同看法的同学,可以立刻离开,我们不讨论。

\section{人才计划导师天团}

熊大,北大光华教授,狗熊会创始人,爱数据爱回归!

水妈,中央财经大学副教授,人才计划的发起人,人称灭绝师太!

若暄,狗熊会全职教师,狗熊会研习社负责人,互联网广告方面专家。

政委,西安交通大学副教授,狗熊会颜值担当。研究专长机器学习。

媛子,厦门大学副教授,狗熊会女神,擅长多元统计分析等领域的研究和教学。

小丫,人民大学副教授,高颜值高智商高情商,个人征信领域专家。

静静,人民大学助理教授,全世界都想她,社交网络分析领域的专家。

菲菲,人民大学助理教授,温柔善良,文本分析领域的专家。

雪姨,复旦大学助理教授,人美辈分高,小说读的好。

昱姐,北大光华在读博士,擅长打麻将!

\chapter{白苹风末}\label{wind}

瞎扯几句。

\section{张老爷子}

话说张老爷子写了一首诗:

\begin{quote}
姑苏开遍碧桃时,邂逅河阳女画师。\\
红豆江南留梦影,白苹风末唱秋词。
\end{quote}

\section{彭大将领}

貌似大家都喜欢用白萍风这个意境。又如彭玉麟的对联:

\begin{quote}
凭栏看云影波光,最好是红蓼花疏、白苹秋老;\\
把酒对琼楼玉宇,莫孤负天心月到、水面风来。
\end{quote}

嘿,玛尼玛尼哄。

\cleardoublepage 

\appendix \addcontentsline{toc}{chapter}{\appendixname}


\chapter{余音绕梁}\label{sound}

呐,到这里朕的书差不多写完了,但还有几句话要交待,所以开个附录,再啰嗦几句,各位客官稍安勿躁、扶稳坐好。

\bibliography{book.bib,packages.bib}

\backmatter
\printindex

\end{document}
